\documentclass[a4paper, notitlepage]{article}
\usepackage{fullpage}
\usepackage{listings}
\usepackage{color}
\usepackage{courier}
\usepackage{graphicx}
\graphicspath{{./pictures/}}
 
\definecolor{dkgreen}{rgb}{0,0.6,0}
\definecolor{gray}{rgb}{0.5,0.5,0.5}
\definecolor{ltgray}{rgb}{0.9,0.9,0.9}
\definecolor{mauve}{rgb}{0.58,0,0.82}
 
\lstset{ %
  language=Java,
  basicstyle=\ttfamily,           % the size of the fonts that are used for the code
  numbers=left,                   % where to put the line-numbers
  numberstyle=\tiny\color{gray},  % the style that is used for the line-numbers
  stepnumber=1,                   % the step between two line-numbers. If it's 1, each line 
                                  % will be numbered
  numbersep=5pt,                  % how far the line-numbers are from the code
  backgroundcolor=\color{ltgray},      % choose the background color. You must add \usepackage{color}
  showspaces=false,               % show spaces adding particular underscores
  showstringspaces=false,         % underline spaces within strings
  showtabs=false,                 % show tabs within strings adding particular underscores
  rulecolor=\color{black},        % if not set, the frame-color may be changed on line-breaks within not-black text (e.g. commens (green here))
  tabsize=2,                      % sets default tabsize to 2 spaces
  captionpos=b,                   % sets the caption-position to bottom
  breaklines=true,                % sets automatic line breaking
  breakatwhitespace=false,        % sets if automatic breaks should only happen at whitespace
  keywordstyle=\color{blue},          % keyword style
  commentstyle=\color{dkgreen},       % comment style
  stringstyle=\color{mauve},         % string literal style
  escapeinside={\%*}{*)},            % if you want to add a comment within your code
}

\begin{document}

\title{IN4331 Web Data Management, assignment 2 \\
Large-Scale Data Management with Hadoop}
\author{Koen Boes (1314785) and Zmitser Zhaleznichenka (4134575)}
\date{\today}
\maketitle

\setcounter{secnumdepth}{0}

\section{19.5.1}

As far as the combiner should do the same as the reducer, we can reuse the reducer code to action like a combiner. This can be done by adding the following code to the \lstinline{AuthorsJob.main()} method.

\begin{lstlisting}
job.setCombinerClass(Authors.CountReducer.class);  
\end{lstlisting}

\section{19.5.2}

The job files are placed in \lstinline{nl.tudelft.in4331.hadoop.ex19_5_2} package. They consist from a mapper, a reducer and a job description.

To split the incoming XML file into separate \lstinline{<movie>} elements, \lstinline{XMLInputFormat} file is used\footnote{http://xmlandhadoop.blogspot.nl}. With this input format, each \lstinline{map()} call gets a string containing XML description of one movie. Later in the mapper this description is parsed with DOM parser and gets deserialized into a \lstinline{Movie} object. The mapper is responsible for emitting two types of key-value pairs for the reducer.

The first pair has key 1 and value needed for "director-and-title" file. The second one respectively has key 2 and value for "title-and-actor" file.

Reducer accepts these pairs and writes the values into two different files using \lstinline{MultipleOutputs} according to the keys with no additional transformations. 

After the job is completed, we get the following contents of the files in Hadoop file system.

\begin{lstlisting}
Unforgiven  Gene Hackman  1930  Little Bill Daggett
A History of Violence  Vigo Mortensen  1958  Tom Stall
A History of Violence  Maria Bello  1967  Eddie Stall
A History of Violence  William Hurt  1950  Richie Cusack
A History of Violence  Ed Harris  1950  Carl Fogarty
Heat  Val Kilmer  1959  Chris Shiherlis
Heat  Robert De Niro  1943  Neil McCauley
Heat  Al Pacino  1940  Lt. Vincent Hanna
Heat  Jon Voight  1938  Nate
Unforgiven  Clint Eastwood  1930  William 'Bill' Munny
Unforgiven  Morgan Freeman  1937  Ned Logan
Match Point  Jonathan Rhys Meyers  1977  Chris Wilton
Match Point  Scarlett  Johansson  1984  Nola Rice
Lost in Translation  Bill Murray  1950  Bob Harris
Lost in Translation  Scarlett  Johansson  1984  Charlotte
Marie Antoinette  Jason  Schwartzman  1980  Louis XVI
Marie Antoinette  Kirsten Dunst  1982  Marie Antoinette
Spider-Man  Tobey Maguire  1975  Spider-Man / Peter Parker
Spider-Man  Willem Dafoe  1955  Green Goblin / Norman Osborn
Spider-Man  Kirsten Dunst  1982  Mary Jane Watson
\end{lstlisting}

\begin{lstlisting}
Michael Mann  Heat  1995
Clint Eastwood  Unforgiven  1992
Sam Raimi  Spider-Man  2002
Sofia Coppola  Lost in Translation  2003
David Cronenberg  A History of Violence  2005
Woody Allen  Match Point  2005
Sofia Coppola  Marie Antoinette  2006
\end{lstlisting}

\section{19.5.3}

Run the following PIG LATIN queries on the files obtained from the previous exercise.

\subsection{19.5.3.1}

Load \textit{title-and-actor.txt} and group on the title. The actors (along with their roles) should appear as a nested bag.

\subsubsection{Code}

\begin{lstlisting}
-- Load records from the title-and-actor.txt file (tab separated).
file = LOAD '../data/title-and-actor.txt' AS (title: chararray, actor: chararray, year_of_birth: int,role: chararray);
-- Group the lines by title.
file_grouped_by_title = GROUP file BY title;
-- Show a subset of the columns (actor, role) in the output.
file_grouped_by_title_subset = FOREACH file_grouped_by_title GENERATE group, file.(actor, role);
-- Output the result.
STORE file_grouped_by_title_subset into '../results/title-and-actor_group-on-title';
\end{lstlisting}

\subsubsection{Result}

\begin{lstlisting}
Heat	{(Al Pacino,Lt. Vincent Hanna),(Jon Voight,Nate),(Val Kilmer,Chris Shiherlis),(Robert De Niro,Neil McCauley)}
Spider-Man	{(Willem Dafoe,Green Goblin / Norman Osborn),(Tobey Maguire,Spider-Man / Peter Parker),(Kirsten Dunst,Mary Jane Watson)}
Unforgiven	{(Gene Hackman,Little Bill Daggett),(Clint Eastwood,William 'Bill' Munny),(Morgan Freeman,Ned Logan)}
Match Point	{(Jonathan Rhys Meyers,Chris Wilton),(Scarlett  Johansson,Nola Rice)}
Marie Antoinette	{(Jason  Schwartzman,Louis XVI),(Kirsten Dunst,Marie Antoinette)}
Lost in Translation	{(Bill Murray,Bob Harris),(Scarlett  Johansson,Charlotte)}
A History of Violence	{(William Hurt,Richie Cusack),(Maria Bello,Eddie Stall),(Ed Harris,Carl Fogarty),(Vigo Mortensen,Tom Stall)}
\end{lstlisting}

\subsection{19.5.3.2}

Load \textit{director-and-title.txt} and group on the director name. Titles should appear as a nested bag.

\subsubsection{Code}

\begin{lstlisting}
-- Load records from the title-and-actor.txt file (tab separated).
file = LOAD '../data/director-and-title.txt' AS (director: chararray, title: chararray, year: int);
-- Group the lines by title.
file_grouped_by_director = GROUP file BY director;
-- Show a subset of the columns (title) in the output.
file_grouped_by_director_subset = FOREACH file_grouped_by_director GENERATE group, file.(title);
-- Output the result.
STORE file_grouped_by_director_subset INTO '../results/director-and-title_group-on-director';
\end{lstlisting}

\subsubsection{Result}

\begin{lstlisting}
Sam Raimi	{(Spider-Man)}
Woody Allen	{(Match Point)}
Michael Mann	{(Heat)}
Sofia Coppola	{(Lost in Translation),(Marie Antoinette)}
Clint Eastwood	{(Unforgiven)}
David Cronenberg	{(A History of Violence)}
\end{lstlisting}

\subsection{19.5.3.3}

Apply the \textbf{cogroup} operator to associate a movie, its director and its actors from both sources.

\subsubsection{Code}

\begin{lstlisting}
-- Load records from the title-and-actor.txt file (tab separated).
file1 = LOAD '../data/director-and-title.txt' AS (director: chararray, title: chararray, year: int);
-- Load records from the title-and-actor.txt file (tab separated).
file2 = LOAD '../data/title-and-actor.txt' AS (title: chararray, actor: chararray, year_of_birth: int,role: chararray);
-- Cogroup file1 and file2
cogrouped = COGROUP file1 BY title, file2 BY title;
-- Show a subset of the columns in the output.
cogrouped_subset = FOREACH cogrouped GENERATE group, file1.(director), file2.(actor, role);
-- Output the result.
STORE cogrouped_subset INTO '../results/director-title-actor_cogroup';
\end{lstlisting}

\subsubsection{Result}

\begin{lstlisting}
Heat	{(Michael Mann)}	{(Al Pacino,Lt. Vincent Hanna),(Jon Voight,Nate),(Val Kilmer,Chris Shiherlis),(Robert De Niro,Neil McCauley)}
Spider-Man	{(Sam Raimi)}	{(Willem Dafoe,Green Goblin / Norman Osborn),(Tobey Maguire,Spider-Man / Peter Parker),(Kirsten Dunst,Mary Jane Watson)}
Unforgiven	{(Clint Eastwood)}	{(Gene Hackman,Little Bill Daggett),(Clint Eastwood,William 'Bill' Munny),(Morgan Freeman,Ned Logan)}
Match Point	{(Woody Allen)}	{(Jonathan Rhys Meyers,Chris Wilton),(Scarlett  Johansson,Nola Rice)}
Marie Antoinette	{(Sofia Coppola)}	{(Jason  Schwartzman,Louis XVI),(Kirsten Dunst,Marie Antoinette)}
Lost in Translation	{(Sofia Coppola)}	{(Bill Murray,Bob Harris),(Scarlett  Johansson,Charlotte)}
A History of Violence	{(David Cronenberg)}	{(William Hurt,Richie Cusack),(Maria Bello,Eddie Stall),(Ed Harris,Carl Fogarty),(Vigo Mortensen,Tom Stall)}
\end{lstlisting}

\subsection{19.5.3.4}

Write a PIG program that retrieves the actors that are also director of some movie: output a tuple for each artist, with two nested bags, one with the movies s/he played a role in, and one with the movies s/he directed.

\subsubsection{Code}

\begin{lstlisting}
-- Load records from the title-and-actor.txt file (tab separated).
file1 = LOAD '../data/director-and-title.txt' AS (director: chararray, title: chararray, year: int);
-- Load records from the title-and-actor.txt file (tab separated).
file2 = LOAD '../data/title-and-actor.txt' AS (title: chararray, actor: chararray, year_of_birth: int,role: chararray);

-- Cogroup file1 and file2
cogrouped = COGROUP file1 BY director, file2 BY actor;

-- Generate a subset of the columns in the output.
cogrouped_subset = FOREACH cogrouped GENERATE group, file2.(title), file1.(title);

-- Filter out actors that didn't direct movies.
cogrouped_subset_filtered = FILTER cogrouped_subset BY NOT (IsEmpty($1) OR IsEmpty($2));

-- Output the result.
STORE cogrouped_subset_filtered INTO '../results/actor_that_is_also_director_of_a_movie';
\end{lstlisting}

\subsubsection{Result}

\begin{lstlisting}
Clint Eastwood	{(Unforgiven)}	{(Unforgiven)}
\end{lstlisting}

\subsection{19.5.3.5}

Write a modified version that looks for artists that were both actors and director of a same movie.

\subsubsection{Code}

\begin{lstlisting}
-- Load records from the title-and-actor.txt file (tab separated).
file1 = LOAD '../data/director-and-title.txt' AS (director: chararray, title: chararray, year: int);
-- Load records from the title-and-actor.txt file (tab separated).
file2 = LOAD '../data/title-and-actor.txt' AS (title: chararray, actor: chararray, year_of_birth: int,role: chararray);

-- Cogroup file1 and file2
cogrouped = COGROUP file1 BY director, file2 BY actor;

-- Generate a subset of the columns in the output.
cogrouped_subset = FOREACH cogrouped GENERATE group, file2.(title), file1.(title);

-- Join both file1 and file2 over the title.
joined = JOIN file1 by title, file2 by title;

-- Take only the actor and director fields.
joined_subset = FOREACH joined GENERATE file1::director, file2::actor;

-- Filter out tuple for which the actor in not equal to the director.
joined_subset_filtered = FILTER joined_subset BY file1::director == file2::actor;

-- Take only the director name (equal to actor name).
actor_that_is_director = FOREACH joined_subset_filtered GENERATE file1::director;

-- Output the result.
STORE actor_that_is_director INTO '../results/actor_that_is_also_director_in_same_movie';
\end{lstlisting}

\subsubsection{Result}

\begin{lstlisting}
Clint Eastwood
\end{lstlisting}

\section{19.5.4}

For this project we decided to work with \lstinline{movies.xml} file as it is and parse it the same way it was done in exercise 19.5.2. We used the movies' summaries as the data source and created movie IDs to distinguish the movies in the resulting files. The IDs were generated automatically by hashing the titles of the movies.

We have created two different jobs having different reducers but sharing the same mapper to complete the assignment. These files are located in \lstinline{nl.tudelft.in4331.hadoop.ex19_5_4} package. We use \lstinline{NormalizationMapper} to parse the data file to extract the summaries. Then, we split the summary text into the tokens based on the white spaces and remove non-alphanumeric symbols not to introduce noise to the results. The mapper emits the current term as the key and a \lstinline{Writable} tuple with \lstinline{movieId} being the first element and serialized list of token positions in the text being the second element. To find the term frequencies, we use \lstinline{NormalizationReducer} class that counts numbers of positions of the terms coming from the mapper and \lstinline{NormalizationJob} that launches the computation. Normalization returns the results in the following format.

\begin{lstlisting}
a	18
about	1
abused	1
after	1
aging	1
ago	1
all	2
an	1
and	22
antoinette	1
antonia	1
anyone	1
anything	1
archduchess	1
are	2
armored	1
around	1
as	4
at	4
attracts	1
aunt	1
austria	2
banks	1
based	1
be	1
been	1
between	2
big	3
bill	1
billing	1
bills	1
birth	1
bitten	1
...
\end{lstlisting}

To output the documents containing given terms and IDF values for the terms, we use the same mapper as above but work with another reducer, \lstinline{DocLevelIndexReducer}. The job description is provided in \lstinline{DocLevelIndexJob} class. It returns the following results.

\begin{lstlisting}
a	-3|16|18|22|38 (1.2)
about	17 (6.0)
abused	17 (6.0)
after	16 (6.0)
aging	-3 (6.0)
ago	22 (6.0)
all	38 (6.0)
an	18 (6.0)
and	-3|16|17|18|22|38 (1.0)
antoinette	17 (6.0)
antonia	17 (6.0)
anyone	38 (6.0)
anything	22 (6.0)
archduchess	17 (6.0)
are	22 (6.0)
armored	22 (6.0)
around	22 (6.0)
as	-3|16|18|38 (1.5)
at	-3|18|22|38 (1.5)
attracts	-3 (6.0)
aunt	16 (6.0)
austria	17 (6.0)
banks	22 (6.0)
based	17 (6.0)
be	18 (6.0)
been	-3 (6.0)
between	-3|18 (3.0)
big	-3|22 (3.0)
bill	-3 (6.0)
billing	-3 (6.0)
bills	-3 (6.0)
birth	17 (6.0)
bitten	16 (6.0)
...
\end{lstlisting} 

Movie IDs are separated by the vertical lines and IDF value is provided between the parentheses.

\end{document}
